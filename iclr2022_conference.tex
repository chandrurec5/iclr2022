\documentclass{article} % For LaTeX2e
\usepackage{iclr2022_conference,times}
\usepackage{hyperref}
\usepackage{url}
\input{pack}
% Optional math commands from https://github.com/goodfeli/dlbook_notation.
\input{math_commands.tex}





\title{Duality simplifies deep neural networks with rectified linear units}

% Authors must not appear in the submitted version. They should be hidden
% as long as the \iclrfinalcopy macro remains commented out below.
% Non-anonymous submissions will be rejected without review.

\author{Antiquus S.~Hippocampus, Natalia Cerebro \& Amelie P. Amygdale \thanks{ Use footnote for providing further information
about author (webpage, alternative address)---\emph{not} for acknowledging
funding agencies.  Funding acknowledgements go at the end of the paper.} \\
Department of Computer Science\\
Cranberry-Lemon University\\
Pittsburgh, PA 15213, USA \\
\texttt{\{hippo,brain,jen\}@cs.cranberry-lemon.edu} \\
\And
Ji Q. Ren \& Yevgeny LeNet \\
Department of Computational Neuroscience \\
University of the Witwatersrand \\
Joburg, South Africa \\
\texttt{\{robot,net\}@wits.ac.za} \\
\AND
Coauthor \\
Affiliation \\
Address \\
\texttt{email}
}

% The \author macro works with any number of authors. There are two commands
% used to separate the names and addresses of multiple authors: \And and \AND.
%
% Using \And between authors leaves it to \LaTeX{} to determine where to break
% the lines. Using \AND forces a linebreak at that point. So, if \LaTeX{}
% puts 3 of 4 authors names on the first line, and the last on the second
% line, try using \AND instead of \And before the third author name.

\newcommand{\fix}{\marginpar{FIX}}
\newcommand{\new}{\marginpar{NEW}}

%\iclrfinalcopy % Uncomment for camera-ready version, but NOT for submission.
\begin{document}



\maketitle

%\begin{comment}
\begin{abstract}
Despite their success deep neural networks (DNNs) are still largely considered as black boxes. The main issue is that the linear and non-linear operations are entangled in every layer, making it hard to interpret the hidden layer outputs. In this paper, we look at DNNs with rectified linear units (ReLUs), and focus on the gating property (`on/off' states) of the ReLUs. We extend the recently developed dual view in which the computation is broken path-wise to show that learning in the gates is more crucial, and learning the weights given the gates is characterised analytically via the so called \emph{neural path kernel} (NPK) which depends on inputs and gates. In this paper, we present novel results to show that convolution with global pooling and skip connection provide rotational invariance and ensemble structure to NPK respectively. To address `black box'-ness, we propose a novel interpretable counterpart of DNNs with ReLUs namely deep linearly gated networks (DLGN): the pre-activations to the gates are generated by a deep linear network, and the gates are then applied as external masks to learn the weights in a different network. The DLGN is not an alternative architecture per se, but a disentanglement and an interpretable re-arrangement of the computations in a DNN with ReLUs. The DLGN disentangles the computations into two  `mathematically' interpretable linearities (i) the `primal' linearity between the input and the pre-activations in the gating network and (ii) the `dual' linearity in the path space in the weights network characterised by the NPK. We compare the performance of DNN, DGN and DLGN on CIFAR-10 and CIFAR-100 to show that, the DLGN recovers more than $83.5\%$ of the performance of state-of-the-art DNNs, i.e., while entanglement in the DNNs enable their improved performance,  the `disentangled and interpretable'  computations in the DLGN can still recover most part of the performance. This brings us to an interesting question: `Is DLGN a universal spectral approximator?'%Finally, we use dual view to show that the degradation in the gates is the reason for degradation in the test performance due to upstream training with random labels (this was an open question in \cite{randlabel}).
%the DLGN counterparts of state of the art DNNs recover more than $83.5\%$ of the performance of the DNNs, which implies that while entanglement in the DNNs enable their improved performance,\
 \end{abstract}
%\end{comment}


\begin{comment}
\begin{abstract}
Despite their success deep neural networks (DNNs) are still largely considered as black boxes. The main issue is that the linear and non-linear operations are entangled in every layer, making it hard to interpret the hidden layer outputs. In this paper, we look at DNNs with rectified linear units (ReLUs), and focus on the gating property (`on/off' states) of the ReLUs. We extend the recently developed dual view in which the computation is broken path-wise to show that learning in the gates is more crucial, and learning the weights given the gates is characterised analytically via the so called \emph{neural path kernel} (NPK) which depends on inputs and gates. In this paper, we first extend the dual view to show that convolution with global pooling and skip connection provide rotational invariance and ensemble structure to NPK respectively. In order to address the issue of `black box'-ness, we propose a novel interpretable counterpart of DNNs with ReLUs namely deep linearly gated networks (DLGN): the pre-activations to the gates are generated by a deep linear network, and the gates are then applied as external masks to learn the weights in a different network. The DLGN disentangles the computations into two  `mathematically' interpretable linearities (i) the `primal' linearity between the input and the pre-activations, and (ii) the `dual' linearity in the path space interpreted via the NPK, and the role of gating is to \emph{lift} the primal to the dual. Our experiments on CIFAR-10 and CIFAR-100 show that the `disentangled and interpretable'  computations in the DLGN can recover more than $83.5\%$ of the performance of the DNNs. 
%the DLGN counterparts of state of the art DNNs recover more than $83.5\%$ of the performance of the DNNs, which implies that while entanglement in the DNNs enable their improved performance,\
 \end{abstract}


\begin{abstract}
We consider DNNs with rectified linear units (ReLUs). We focus on the gates, i.e., `on/off' states of the ReLUs. We build on dual view of \cite{npk} in which the computation is broken path-wise, where each path has both gates and weights. They showed that the input dependent gating patterns act like features, and characterised them analytically via the so called \emph{neural path kernel} (NPK) which depends on number of paths simultaneously `on' for a given input pair. New results in this paper deal with: (i) uncovering structural properties of NPK, (ii) demonstrating `layer-path' duality (iii) resolving an open question on training with random labels and (i) building comptetitve white box models.
%Firstly, we present new results that simplify and improve our understanding of DNNs with ReLUs. For this, we present three new theoretical results, using which we interepret the roles of activations, weights, width, depth, convolutions with global pooling and skip connections. 
Firstly, we show that the NPK has a product structure invariant to layer permutation and only depends on the correlation of the gates. We show that convolution with global pooling and skip connection provide rotational invariance and ensemble structure to NPK respectively. Secondly, we show that once the gates are obtained in a layer-by-layer manner, operations destroying the layer-by-layer structure such as permuting the layers, arbitrarily tiling and rotation of the gates and providing a constant input do not degrade performance, because, in all these operations, the correlation of gates is not lost. Thirdly, we use the dual view to resolve the open question related to degradation of test accuracy due to upstream training with random labels : degradation occurs because random labels affect the gates. Finally, we propose a novel model by modifying  DNNs with ReLUs: pre-activations to the gates are generated without any non-linear activations. Interpreted in the dual view, this novel model is entirely white box. We show white box models obtained by modifying VGG and a ResNet (in the proposed way) achieve greater than $90\%$ and close to $70\%$ on CIFAR-10 and CIFAR-100 respectively.  
%We present an improved and simplified understanding of deep neural network (DNNs) with rectified linear units (ReLUs), and use our understanding to build white box models. We focus on the gates, i.e., `on/off' states of the ReLUs. We build on prior work by \cite{npk} developed a dual view to showed that most information is in the gates, and captured the role of the gates analytically via the so called \emph{neural path kernel} (NPK). 
%In this paper, we show that interpreting via the dual view offers novel, fundamental, surprising and counter-intutive results eventually paving way for building white box models. %We achieve our aim by persuing two goals (i) conceptual: where our key objective is improved understanding and not necessarily to beat state of the art  and (ii) practical: where our key objective is to improve interpretability without significant loss with respect to state of the art. 
%FThe conceptual goal is to drive home the DNNs are interpretable in the dual view. 
%We reach our conceptual goal in two steps. 
%Firstly, we present new theoretical results by refining the prior work on dual view for the fully connected case, and extending it to cover the case of convolutions with global pooling and skip connections. The main highlight is that each layer has a base kernel measuring the correlation of gates, and the NPK is a Hadamard product of these base kernels and the input Gram matrix. We experimentally verify that operations destroying the layer-by-layer structure such as permuting the layers, arbitrarily tiling and rotation of the gates and providing a constant input do not degrade performance, because, in all these operations, the correlation of gates is not lost. We also show that upstream training with random labels degrades the gates and hence test accuracy even after downstream training with true labels (this was an open question). Finally, we propose a novel model by modifying  DNNs with ReLUs: we generate the pre-activation to the gates without any non-linear activations. Interpreted in the dual view, this novel model is entirely white box. We show white box models obtained by modifying VGG and a ResNet (in the proposed way) achieve greater than $90\%$ and close to $70\%$ on CIFAR-10 and CIFAR-100 respectively. %The empirical results here are of two kinds (i) conceptual: the key objective is improved understanding by verifying the theory and not necessarily to beat `state of the art' and (ii) practical: the key objective is to build white box models without significant loss with respect to `state of the art'.
%We present an improved and simplified understanding of  deep neural network (DNNs) with rectified linear units (ReLUs). In particular, we focus on the gating property (i.e., \emph{on/off} state) of ReLU, due to which, for each input there is an \emph{active/on} sub-network comprising of those gates which are \emph{on} and the weights between those gates. Recently, \cite{npk} developed a \emph{dual view} to separate the gates and the weights. They showed that most information is in the gates, and captured the role of the active sub-networks analytically via the so called \emph{neural path kernel} (NPK). In this paper, we simplify the NPK by expressing it explicitly in terms of the \emph{correlation of gates}, and derive the additional properties of NPK in the presence of convolutions and skip connections. The main highlight is that each layer has a base kernel measuring the correlation of gates, and the NPK is a Hadamard product of these base kernels and the input Gram matrix.  These show that the dual view is a natural way to think about the inner workings of DNNs with ReLUs. Finally, we propose a novel deep network (we call this \texttt{DGN-No-ACT}) wherein the gates are generated without any non-linear activations. This makes \texttt{DGN-No-ACT} a completely white box model. We show that \texttt{DGN-No-ACT} based on standard architectures achieve more than $90\%$ and close to $70\%$ on CIFAR-10 and CIFAR-100 respectively.


%Finally, we modify standard architectures (VGG19 and a ResNet) to yield two deep gated networks in which feature extraction is free of activations and is separate from the gates and the weights -- these achieve greater than $90\%$ test accuracy on CIFAR-10. 

%The other two theoretical results extend the dual view to cover the cases of convolutions with pooling and skip connections.
%Our main message is that the gates hold most useful information. 
%We present a simplified and improved understanding of the inner workings of deep neural networks (DNNs) with rectified linear units (ReLUs) by focussing on the gating (i.e., `on/off' states) of the ReLUs. We build on prior work by \cite{npk} which also focussed on the role of gates in DNNs with ReLUs.  Our main claim is that gates are indeed the most fundamental entities in such DNNs that hold most useful information. We provide theoretical basis for the claim and experimental justification. Based on this simplified understanding, we conceptualise a DNN with ReLU to have three functional components (i) gating, (ii) pre-activation generation and (iii) weights.  In a  DNN with ReLU these three functionalities are shared/entangled between its weights and activation. We propose a novel modification to disentangle these three components thereby making the deep network entirely interpretable. We show that applying this modification on standard state of the art DNNs makes them entirely interpretable without significant loss of performance.
%we propose a novel modification that yields an entirely interpretable deep network.
 
%We present an improved and simplified understanding of  deep neural network (DNNs) with rectified linear units (ReLUs) by focussing on the gating property (i.e., on/off state) of ReLU.  We build on the \emph{dual view} introduced by \cite{npk}. The key simplification is the claim that DNNs with ReLUs are characterised by the \emph{correlation of gates}. We verify this claim by showing that operations destroying the layer-by-layer structure such as permuting the layers, arbitrarily tiling and rotation of the gates and providing a constant input do not degrade performance, because, in all these operations, the correlation of gates is not lost. We then take up an open question related to the degradation of test accuracy due to upstream training with random labels for study. Using the dual view, we show that this degradation is attributed to the gates thereby demonstrating the importance of the role of gates and efficacy of the dual view in understanding DNNs with ReLUs. Based on our improved understanding, we propose a novel modification that improves `interpretability' :  here (i) feature extraction, (ii) gating and (iii) weights are decoupled. We show on standard architectures that this novel modification achieves greater than $90\%$ and close to $70\%$ test accuracies on CIFAR-10 and CIFAR-100 respectively while improving interpretability.

%We present an improved and simplified understanding of  deep neural network (DNNs) with rectified linear units (ReLUs). We build upon the dual view developed by \cite{npk}, wherein the computations are broken path-by-path as opposed to the primal view where computations are layer-by-layer. In the dual view, the output is expressed as an inner product of so called \emph{neural path features} (which encodes the input dependent computation) and so called \emph{neural path values} (which encode computations common across inputs). Our theoretical results refine the prior dual view for fully connected case, and extend it to cover the cases of convolutions with global pooling and skip connections. We also present empirical results which, are open and counter intuitive from the primal view, however, are reconciled in the dual view. These results suggest that dual view is the natural way to interpret DNNs with ReLUs. Finally, we propose a novel modified deep network wherein the neural path features are generated without using any non-linear activations (which call this network \texttt{DGN-No-ACT}). Here, features are not hidden, and are entirely interpretable in terms of standard `image processing' operations. This makes \texttt{DGN-No-ACT} a completely white box model. We show that \texttt{DGN-No-ACT} based on standard architectures achieve more than $90\%$ and close to $70\%$ on CIFAR-10 and CIFAR-100 respectively.

%We present an improved and simplified understanding of  deep neural network (DNNs) with rectified linear units (ReLUs) by focussing on the gating (i.e., `active/inactive' states) of the ReLUs. We build upon the dual view wherein the computation in a DNN is broken down into paths. Prior work on dual view (for fully connected case) showed that the most useful information in the gates, and is characterised by a so called \emph{neural path kernel} (NPK) which depends on the total number of active paths. Our theoretical results refine the prior dual view for fully connected case, and extend it to cover the cases of convolutions with global pooling and skip connections. The main highlight is that each layer has a base kernel measuring the correlation of gates, and the NPK is a Hadamard product of these base kernels and the input Gram matrix. To our knowledge, this is the simplest kernel in literature that analytically characterises the gates. We experimentally verify that operations destroying the layer-by-layer structure such as permuting the layers, arbitrarily tiling and rotation of the gates and providing a constant input do not degrade performance, because, in all these operations, the correlation of gates is not lost. We also show experimentally that (open question related to) the degradation in test accuracy due to upstream training with random labels is because of the degradation to gates, thereby reinforcing the importance of gates. %We also present empirical results which, are open and counter intuitive from the primal view, however, are reconciled in the dual view. These results suggest that dual view is the natural way to interpret DNNs with ReLUs. 
%Finally, we propose a novel modified deep network wherein the gates are generated without using any non-linear activations (which call this network \texttt{DGN-No-ACT}). Here, features are not hidden, and are entirely interpretable in terms of standard `image processing' operations. This makes \texttt{DGN-No-ACT} a completely white box model. We show that \texttt{DGN-No-ACT} based on standard architectures achieve more than $90\%$ and close to $70\%$ on CIFAR-10 and CIFAR-100 respectively.
%for a given pair of inputs is equal to the product of the inner product of the inputs and the sizes of the overlapping active sub-networks corresponding to the inputs.
\end{abstract}
\end{comment}







%Despite their success deep neural networks (DNNs) are still largely considered as black boxes. In this paper, we present a simplified and improved understanding of DNNs with rectified linear units (ReLUs). Unlike several methods which ‘explain’ the inner workings of DNNs by building simpler local models, our approach is to ‘dismantle’ the DNN with ReLU into separate components. In particular, we focus on the gating property (i.e., on/off states) of the ReLUs. We build on the prior work by Lakshminarayanan & Singh (2020), who developed a novel dual view, wherein the gates are dismantled from the weights. In this paper, we carry forward this dismantling, where the gates in a DNN are dismantled layer-by-layer, and finally the gates in a layer are dismantled unit-by-unit. The key simplification is in our main claim that gates are indeed the most fundamental entities in DNNs with ReLUs. We provide theoretical basis for this claim and justify the same in experiments. Based on our theory and experiments, we argue that a DNN with ReLU into three functionalities namely (i) gating (ii) pre-activation generation and (iii) weights. While a standard DNN with ReLU is such that these three functionalities are shared/entangled between its weights and activation, we propose a novel modification wherein we disentangle the three components.



\section{Introduction}

Despite their success deep neural networks (DNNs) are still largely considered as black boxes. In this paper, we present a simplified and improved understanding of DNNs with rectified linear units (ReLUs). Our approach is to \emph{dismantle} a DNN with ReLUs in a step by step manner to identify its \emph{functional components} and understand their role. This is in contrast to approaches that \emph{explain} the \emph{decisions} of a DNN by building simpler local models. Based on our understanding of such functional components, we finally propose a novel way to put them together in an entirely interpretable way thereby removing the `black box'-ness. Our work builds on top of prior work by \cite{npk} who developed a \emph{dual view} for fully connected DNNs with ReLUs.


\subsection{Dual View: First step in understanding the role of gates}

The dual view exploits the gating property of ReLU, that is, the \emph{on/off} or \emph{active/inactive} states of the ReLUs. It is natural to think that depending on the input, some subset of the gates get activated, and computation from input to output is restricted to the `active' subnetwork consisting of such active gates and the weight through such gates. Thus, the gates are input dependent and the weights remain the same across inputs. However, weights are triggering the gates in the first place. So, in order to separately the understand of their roles, the gates are also physically separated from the weights by generating the input dependent gates in a so called \emph{feature network} (the sole purpose of whose weights is to trigger the gates) and then applying gating signals from the feature network as external masks to a so called \emph{value network} which carries out the input to output computation. Having separated the gates and the weights, the role of gates was investigates in theory and experiments.

\textbf{Theory (Neural Path Kernel).} A \emph{neural path kernel} (NPK) is defined and is equal to the \emph{Hadamard} product of the input Gram matrix and a correlation matrix which measures the overlap between the sub-networks active for the various input pairs. Prior results  \cite{arora2019exact,cao2019generalization,ntk} have shown the equivalence of an infinite width DNN trained using gradient descent and its corresponding \emph{neural tangent kernel} (NTK) matrix, the Gram matrix of the gradient of the network output with respect to the weights. It was shown that when the gates and weights are separated, with the gates being fixed and only the weights trained, under randomised initialisation, in the limit of infinite width, the NTK becomes equal to (but for a scalar term) the NPK. This equivalence between NTK and NPK analytically characterises the role of gates in terms of the active subnetworks.


\textbf{Experiments (Gate Learning).}  It was shown that by using the gates from a pre-trained DNN as external masks one can retrain the weights and match the test accuracy of the original pre-trained DNN with ReLU. 
This implies that most useful input dependent information is in the gates. It was shown that if gates from a randomly initialised (instead of pre-trained) DNN are used as external masks, and the weights are trained, the test accuracy drops significantly. This implies that gates are learnt when a DNN with ReLU is trained and such learning improves generalisation.

\textbf{Significance of Duality.} 

\subsection{Contribution and Organisation}
 
In this paper, we first build on the dual view,

\textbf{\Cref{sec:fc} : Theory (Simplified NPK).}  We understand the roles of the various functional components. First, significant results is \Cref{th:} a simple kernel expression (for the fully connected case) which the roles of (i) activations, (ii) weight, (ii) width, (ii) depth. 
 (iii) convolutions with pooling and (iv) skip connections. Firstly, we 

In the fully connected case, the key message is that each layer has an associated base kernel which measures the \emph{average} 
we show that the NPK is a \emph{Hadamard} product of the input Gram matrix and $$

talk about functional components


\textbf{Experiments (Gate Learning).} 

Based on our improved understanding we finally identify three functional components namely (i) pre-activation generation to trigger the gates (ii) gating to select the input dependent sub-network and (iii) output computation using weights in the sub-network. We propose a deep gated network architecture wherein these three are physically separated. 




In this paper, we carry forward this dismantling, where the gates in a DNN are dismantled layer-by-layer, and finally the gates in a layer are dismantled unit-by-unit. The key simplification is in our main claim that gates are indeed the most fundamental entities in DNNs with ReLUs. We provide theoretical basis for this claim and justify the same in experiments. Based on our theory and experiments, we argue that a DNN with ReLU into three functionalities namely (i) gating (ii) pre-activation generation and (iii) weights. While a standard DNN with ReLU is such that these three functionalities are shared/entangled between its weights and activation,  we propose a novel modification wherein we disentangle the three components. 

 separate the functionalities to dedicated for (i) feature generation (without any hidden units), (ii) gating, and (iii) weights. 

Due to this separation of functionalities network is entirely interpretable by design. Now, we first summarise the main results in the prior work by \cite{npk} who developed a dual view to understand the role of gates in DNNs with ReLUs, followed by the specific contributions in this paper.





%Our main claim is that gates are indeed the most fundamental entities in such DNNs. We provide theoretical basis for the claim and justify the same in experiments. Based on this claim, we propose a novel modification wherein the deep network has separate components to dedicated specifically to address (i) feature generation (without any hidden units), (ii) gating, and (iii) the weights in a decoupled manner. Due to this decoupled network is entirely interpretable by design. Now, we first summarise the main results in the prior work by \cite{npk} who developed a dual view to understand the role of gates in DNNs with ReLUs, followed by the specific contributions in this paper.

\subsection{Background: Dual View for DNNs with ReLUs}

A special property of a ReLU is that it is also a gate (i.e., `on/off') which either allows (i.e., multiplies by 1 in ‘on’ state) or blocks (i.e., multiplies by 0 in ‘off’ state) its pre-activation input based on positivity of the same. The gating property naturally gives rise to a ‘sub-network’ based interpretation of DNNs with ReLUs: for each input, there is a corresponding set of ReLUs that are active (in ‘on’ state) and these ReLUs together with the weights that connect them form the ‘active’ sub-network that is responsible for producing the output for that input.


We provide theoretical basis for the claim and justify the same in experiments. Based on this claim, we propose a novel modification wherein the deep network has separate components to dedicated specifically to address (i) feature generation (without any hidden units), (ii) gating, and (iii) the weights in a decoupled manner. Due to this decoupled network is entirely interpretable by design. 

In the dual view, \cite{npk} exploited the gating property of ReLU to break the DNN into paths, where each path comprises of gates and weights. This allows for separation of gates from weights, in that, the gates are treated as masks and are decoupled from the weights by storing the gates and weights in two separate networks. The information in the gates is then measured by fixing the gates, training only the weights and looking at the test performance. In this fixed gate setting, their main theoretical result is that the information stored in the gates is characterised by a so called \emph{neural path kernel} which for given inputs $x,x'$: (i) is equal (but for a scaling factor) to the \emph{neural tangent kernel} (NTK), (ii) is equal to product of the inner product of the inputs and the size of the `active' sub-network overlapping/common for both inputs.
\begin{align}\label{eq:ntk-npk-relation}
\text{NTK}(x,x')\quad\propto\quad \text{NPK}(x,x')\quad =\quad \ip{x,x'}\cdot {\bf{overlap}}(x,x'),
\end{align}

%\section{Introduction}
Despite their success deep neural networks (DNNs) are still largely considered as black boxes. The larger aim of this paper is to argue that, at least in the case of DNNs with rectified linear units (ReLUs), the issue of `black box'-ness can be addresses successfully by simple a change of perspective/viewpoint.  To elaborate, the issue has less to do with the models themselves. Instead the issue is mostly a conceptual one arising from a commonly held, mathematically correct, yet ill-chosen primal view/perspective. The primal view is that as computation in a DONN proceeds from the input to output in a layer-by-layer manner increasingly sophisticated features are learnt in the hidden layers of the DNN and the final layer learns a linear model with the hidden features output by the penultimate layer. The conceptual issue with the primal view is that  while the final layer is linear and amenable to a feature/weight interpretation, however, the hidden features obtained as a result of several non-linear operations on the input are not understandable themselves. Recently \cite{npk} used the \emph{dual view} to investigate the role of gates (i.e., \emph{on/off} state) of the ReLUs. We argue that by switching over to the dual view the `black box'-ness issue vanishes.


The dual view decomposes the computation in a DNN into two parts (i) computation in the gates which is input dependent and (ii) computation in the weights which is the same across inputs. Based on this decomposition, the gates are separated from the weights. The information in the gates is then characterised by keeping the gates fixed, applying them as external masks and then training the weights. In this setting, the following interesting results were shown:

1. \textbf{ Most information is in the gates}. Using the gates from a pre-trained DNN as external masks one can retrain the weights and match the test accuracy of the original pre-trained DNN with ReLU. 

2. \textbf{Gate Are Learnt.} It was shown that gates are learnt during training and it improves test accuracy. It was shown that if gates from a randomly initialised (instead of pre-trained) DNN are used as external masks, and the weights are trained, the test accuracy drops significantly. This implies that gates are learnt when a DNN with ReLU is trained.
  
%This implies that most useful input dependent information is in the gates. 
3. \textbf{Neural Path Kernel For Subnetwork.} For any given input, computation from input to output happens via the `active subnetwork' consisting of active paths.
It was show that the information stored in the gates is analytically characterised by \emph{neural path kernel} (NPK) which is equal to the \emph{Hadamard} product of the input Gram matrix and a correlation matrix which measures the overlap between the sub-networks active for the input pairs.

\subsection{Contribution and Organisation}

In this paper, we show that the dual view serves as a conceptual useful tool to interpret and understand the inner workings of DNNs with ReLU and also it serves as a practical tool that leads to design of white box models. To this end we pursue the following two goals:

\emph{Part I: Componentwise understanding of DNNs with ReLUs.}
 
Using the dual view, we interpret the roles of (i) width, (ii) depth, (iii) convolutions with global pooling and (iv) skip connections. For this, we refine and extend prior work to derive new results on the structure of NPK. We then present two new empirical results that are easily reconciled in the dual view (as opposed to the primal). These results are not meant to be techniques/tricks to improve performance that `beats state of the art'. The significance is that they showcase the power of dual view in resolving novel and unseen scenarios. We now list these contributions section-wise.

$\bullet$ In \Cref{sec:theory}, we refine prior result on NPK (which depended on the correlation subnetworks) to show that NPK depends on the \emph{correlation of gates}. The main highlight is that each layer is associated with a base kernel which measures the average correlation of gates, and the NPK is a \emph{Hadamard} product of the input Gram matrix and the base kernels. This implies that the role of width is averaging and the role of depth is to form a product kernel. In \Cref{th:conv} we show that the role of convolutions with global pooling is to provide rotational invariance to the NPK. In \Cref{th:res} we show that the role of skip connections is to provide an ensemble structure to NPK.


$\bullet$ In \Cref{sec:permute}, we present surprising results that are counter intuitive with respect to the primal view that progressively sophisticated structures are being in a layer by layer manner. Firstly, we show that destroying the layer by layer structure by permuting the gates does not cause performance to degrade at all. Secondly, we show that once the gates are obtained from the input, the network can be provided with a constant $\mathbf{1}$ input without degrading performance. We argue that these counter intuitive results are easily reconciled in the dual view.

\emph{Part II: Open question related to training with random labels.}

 In \Cref{sec:randlabel}, we show that upstream training with random labels followed by downstream training with true labels degrades test accuracy because the random labels affects the gates. This degradation of test accuracy was an open question in \cite{randlabel}.


\emph{Part III: Entirely interpretable and white box model.}

In \Cref{sec:whitebox}, we propose a novel model by modifying  DNNs with ReLUs: we generate the pre-activation to the gates without any non-linear activations. Interpreted in the dual view, this novel model is entirely white box. We show white box models obtained by modifying VGG and a ResNet (in the proposed way) achieve greater than $90\%$ and close to $70\%$ on CIFAR-10 and CIFAR-100 respectively. 


%we propose a novel architecutre obtained by modifying DNNs with ReLUs, which, is conceptually same as DNNs with ReLUs but is entirely interpretable and white box. We effect the modification on a stardard model namely VGG and a ResNet model, to derive the corresponding white box models. We show these white box models achieve greater than $90\%$ and close to $70\%$ on CIFAR-10 and CIFAR-100 respectively. Here, the aim is to demonstrate that the new white box models improve interpretablity without significant loss in performance with respect to `state of the art'.


%We follow up this claim by completely removing the `black box'-n ess: we propose a novel architecutre, which, is conceptually same as DNNs with ReLUs but is entirely interpretable and white box. 
\begin{comment}
\subsection{Dual View}
In the dual view, the computation in a DNN is broken down into paths, wherein, each path starts from the input node, passes through a weight and a ReLU in each layer and ends at the output node. This gives a subnetwork intepretation: for any given input, computation from input to output happens via the `active subnetwork' consisting of active paths. This active subnetwork is in turn formed by the active gates in each layer. Each gate is a just a  simple `perceptron' which is described by the hyperplane of its incoming weights. As the the input passes through the layers, gates are turned on/off based on the angle between the layer input and the hyperplanes associated with the various gates in that layer. 


 The separation is achieved by a \emph{deep gated network} (DGN) setup, wherein, the input dependent gates are generated in a so called \emph{feature network} (which a DNN with ReLU whose sole  is  generation of the gates) and then gating signals are applied as external masks to a so called \emph{value network} which carries out the input to output computation. Having separated the gates and the weights, the role of gates and active subnetworks was investigated in theory and experiments.

%The main result  It was shown that most useful information is in the gates.  

%The dual view (i) gives simpler feature/value decomposition, (ii) gives a subnetwork intepretation and (iii) allows to separate the gates from the weights. We briefly described these below.

%Each path has a signal at its input node, which is gets transmitted to the output if the path is on (i.e., all the gates in the paths are on). On its way to the output, (if path is active) the input signal of a path is scaled by a `value' equal the product of the weights in the path. The input and the on/off of the path is encoded in a so called  and the scaling by the weights is encoded by the \emph{neural path value} (NPV $\in \R^{\text{total-paths}}$). This conceptual separation of computation in the gates encoded in the NPF and the computation in the weights in the NPV has the following favourable points:

%$\bullet$ \textbf{Simplicity.} The input and the gates of a path are  encoded in a \emph{neural path feature} (NPF $\in \R^{\text{total-paths}}$). The weights are encoded in a \emph{neural path value} (NPV $\in \R^{\text{total-paths}}$). The NPF coordinate of a path is $0$ if the path is inactive and is equal to the signal at the input node if the path is active.  The NPV coordinate of a path is the product of the weights in the path. The NPV of a path the scales signal at it input node, and the final DNN output is the inner product of NPF and NPV.

%$\bullet$ \textbf{Interpretability.} For any given input, computation from input to output happens via the `active subnetwork' consisting of active paths. This active subnetwork is in turn formed by the active gates in each layer. Each gate is a `perceptron' which is described by the hyperplane of its incoming weights. As the the input passes through the layers, gates are turned on/off based on the angle between the layer input and the hyperplanes associated with the various gates in that layer. 

%$\bullet$ \textbf{Separation.} 
%In a DNN with ReLUs, the weights play a dual role: (i) they decide the on/off activity of the gates, i.e., the NPF and they also dictate the NPV. 
%To understand the role of gates and weights separately, the gates are also physically separated from the weights in \emph{deep gated network} (DGN) setup, wherein, the input dependent gates are generated in a so called \emph{feature network} (which a DNN with ReLU whose sole  is  generation of the gates) and then gating signals are applied as external masks to a so called \emph{value network} which carries out the input to output computation. Having separated the gates and the weights, the role of gates and active subnetworks was investigated in theory and experiments.

1. \textbf{Gate Learning.} Most information is stored in the gates. Using the gates from a pre-trained DNN as external masks one can retrain the weights of the value network and match the test accuracy of the original pre-trained DNN with ReLU. It was shown that gates are learnt during training and it improves test accuracy.  
%This implies that most useful input dependent information is in the gates. 
%It was shown that if gates from a randomly initialised (instead of pre-trained) DNN are used as external masks, and the weights are trained, the test accuracy drops significantly. 
%This implies that gates are learnt when a DNN with ReLU is trained and

2. \textbf{Neural Path Kernel.} The information stored in the gates is analytically characterised by \emph{neural path kernel} (NPK) which is equal to the \emph{Hadamard} 
\end{comment}
\begin{comment}

\subsection{Dual View}
In the dual view, the computation in a DNN is broken down into paths, wherein, each path starts from the input node, passes through a weight and a ReLU in each layer and ends at the output node. This gives a subnetwork intepretation

 and allows to separate information in the gates from the weights. The main result  It was shown that most useful information is in the gates.  

%The dual view (i) gives simpler feature/value decomposition, (ii) gives a subnetwork intepretation and (iii) allows to separate the gates from the weights. We briefly described these below.

%Each path has a signal at its input node, which is gets transmitted to the output if the path is on (i.e., all the gates in the paths are on). On its way to the output, (if path is active) the input signal of a path is scaled by a `value' equal the product of the weights in the path. The input and the on/off of the path is encoded in a so called  and the scaling by the weights is encoded by the \emph{neural path value} (NPV $\in \R^{\text{total-paths}}$). This conceptual separation of computation in the gates encoded in the NPF and the computation in the weights in the NPV has the following favourable points:

%$\bullet$ \textbf{Simplicity.} The input and the gates of a path are  encoded in a \emph{neural path feature} (NPF $\in \R^{\text{total-paths}}$). The weights are encoded in a \emph{neural path value} (NPV $\in \R^{\text{total-paths}}$). The NPF coordinate of a path is $0$ if the path is inactive and is equal to the signal at the input node if the path is active.  The NPV coordinate of a path is the product of the weights in the path. The NPV of a path the scales signal at it input node, and the final DNN output is the inner product of NPF and NPV.

$\bullet$ \textbf{Interpretability.} For any given input, computation from input to output happens via the `active subnetwork' consisting of active paths. This active subnetwork is in turn formed by the active gates in each layer. Each gate is a `perceptron' which is described by the hyperplane of its incoming weights. As the the input passes through the layers, gates are turned on/off based on the angle between the layer input and the hyperplanes associated with the various gates in that layer. 

$\bullet$ \textbf{Separation.} 
%In a DNN with ReLUs, the weights play a dual role: (i) they decide the on/off activity of the gates, i.e., the NPF and they also dictate the NPV. 
To understand the role of gates and weights separately, the gates are also physically separated from the weights in \emph{deep gated network} (DGN) setup, wherein, the input dependent gates are generated in a so called \emph{feature network} (which a DNN with ReLU whose sole  is  generation of the gates) and then gating signals are applied as external masks to a so called \emph{value network} which carries out the input to output computation. Having separated the gates and the weights, the role of gates and active subnetworks was investigated in theory and experiments.

\textbf{Key Results.}  

1. \textbf{Gate Learning.} Most information is stored in the gates. Using the gates from a pre-trained DNN as external masks one can retrain the weights of the value network and match the test accuracy of the original pre-trained DNN with ReLU. It was shown that gates are learnt during training and it improves test accuracy.  
%This implies that most useful input dependent information is in the gates. 
%It was shown that if gates from a randomly initialised (instead of pre-trained) DNN are used as external masks, and the weights are trained, the test accuracy drops significantly. 
%This implies that gates are learnt when a DNN with ReLU is trained and

2. \textbf{Neural Path Kernel.} The information stored in the gates is analytically characterised by \emph{neural path kernel} (NPK) which is equal to the \emph{Hadamard} product of the input Gram matrix and a correlation matrix which measures the overlap between the sub-networks active for the input pairs.
\end{comment}
\begin{comment}

\textbf{Neural Path Feature.} In the dual view, the computation in a DNN is broken down into paths, which leads to an alternative feature/value decomposition. Each path has a signal at its input node, which is gets transmitted to the output if the path is on (i.e., all the gates in the paths are on). On its way to the output, (if transmitted) the input signal of a path is scaled by a `value' equal the product of the weights in the path. The input and the on/off of the path is encoded in a so called \emph{neural path feature} (NPF $\in \R^{\text{total-paths}}$) and the scaling by the weights is encoded by the \emph{neural path value} (NPV $\in \R^{\text{total-paths}}$). The DNN output for an input $x\in\R^{\din}$ , and parameter $\Theta\in\R^{\dnet}$ is given by:
\begin{align}
\texttt{DNN-OUTPUT(x)=}\ip{\texttt{NPF}_{\Theta}\texttt{(x),NPV}_{\Theta}}
\end{align}

\textbf{Subnetwork Interpretation.} For any given input, the NPF coordinate is zero for all the inactive paths. This provides a subnetwork based interpretation, that is, computation from input to output happens via the `active subnetwork' consisting of active paths.  



As as result, the Gram matrix of the NPFs called the \emph{neural path kernel} (NPK) is equal to the \emph{Hadamard} product of the input Gram matrix and a correlation matrix which measures the overlap between the sub-networks active for the various input pairs. 



\textbf{Self-Explanation.}


For this purpose, the dual view exploits the gating property of ReLU, that is, the \emph{on/off} or \emph{active/inactive} states of the ReLUs. A path starts at an input node, passes through a weight and a ReLU in each layer until it reaches the output node. A path is active all the gates in that path active, and its contribution is equal to the product of the signal at the input node and the weights in the path. In the dual view, it is natural to think that depending on the input, some subset of the paths get activated, and computation from input to output is restricted to the `active' subnetwork consisting of such active paths. Thus, the gates are input dependent and the weights remain the same across inputs. Holwever, weights are triggering the gates in the first place. So, in order to separately the understand of their roles, the gates are also physically separated from the weights in \emph{deep gated network} (DGN) setup, wherein, the input dependent gates are generated in a so called \emph{feature network} (which a DNN with ReLU whose sole  is  generation of the gates) and then gating signals are applied as external masks to a so called \emph{value network} which carries out the input to output computation. Having separated the gates and the weights, the role of gates and active subnetworks was investigated in theory and experiments.

\textbf{Neural Path Kernel (Theory).} A \emph{neural path kernel} (NPK) is defined and is equal to the \emph{Hadamard} product of the input Gram matrix and a correlation matrix which measures the overlap between the sub-networks active for the various input pairs. Prior results  \cite{arora2019exact,cao2019generalization,ntk} have shown the equivalence of an infinite width DNN trained using gradient descent and its corresponding \emph{neural tangent kernel} (NTK) matrix, the Gram matrix of the gradient of the network output with respect to the weights. It was shown that when the gates and weights are separated, with the gates being fixed and only the weights trained, under randomised initialisation, in the limit of infinite width, the NTK becomes equal to (but for a scalar term) the NPK. This equivalence between NTK and NPK analytically characterises the role of the active subnetworks.


\textbf{Gate Learning (Experiments).}  It was shown active subnetworks are learnt during training and it improves test accuracy.  Using the gates from a pre-trained DNN as external masks one can retrain the weights and match the test accuracy of the original pre-trained DNN with ReLU. 
%This implies that most useful input dependent information is in the gates. 
It was shown that if gates from a randomly initialised (instead of pre-trained) DNN are used as external masks, and the weights are trained, the test accuracy drops significantly. 
%This implies that gates are learnt when a DNN with ReLU is trained and such learning improves generalisation.
\end{comment}



%In all results above, the gates were generated by a feature network which was a DNN with ReLUs.  Based on our understanding, we propose to replace the ReLUs with identity activations giving rise to a white box architecture called \texttt{DGN-NO-ACT}. Once the ReLU non-linearity is removed, the other operations such as convolution (which is  linear), pooling and batch norm (bias and scaling) are well understood and interpretable in `image processing' terms. 


%We pursue two goals (i) pedagogical: here the pursuit is not propose new methods to beat the state of the art, but to improve our understanding of basic functional components namely weights, activation, depth, width, convolutions with global pooling  and skip connection (ii) practical: here the pursuit is to build a white box model without significance performance loss with respect to state of the art. The pedagogical goal is to drive home the message that, even though the primal and dual views are mathematically equivalent, when compared to the primal, the dual view is a natural and simple way to interpret and understand the inner workings of DNNs with ReLUs. The pedagogical goal is achieved in the following two steps. 





%In this paper, we carry forward this dismantling, where the gates in a DNN are dismantled layer-by-layer, and finally the gates in a layer are dismantled unit-by-unit. The key simplification is in our main claim that gates are indeed the most fundamental entities in DNNs with ReLUs. We provide theoretical basis for this claim and justify the same in experiments. Based on our theory and experiments, we argue that a DNN with ReLU into three functionalities namely (i) gating (ii) pre-activation generation and (iii) weights. While a standard DNN with ReLU is such that these three functionalities are shared/entangled between its weights and activation,  we propose a novel modification wherein we disentangle the three components. 

% separate the functionalities to dedicated for (i) feature generation (without any hidden units), (ii) gating, and (iii) weights. 

%Due to this separation of functionalities network is entirely interpretable by design. Now, we first summarise the main results in the prior work by \cite{npk} who developed a dual view to understand the role of gates in DNNs with ReLUs, followed by the specific contributions in this paper.





%Our main claim is that gates are indeed the most fundamental entities in such DNNs. We provide theoretical basis for the claim and justify the same in experiments. Based on this claim, we propose a novel modification wherein the deep network has separate components to dedicated specifically to address (i) feature generation (without any hidden units), (ii) gating, and (iii) the weights in a decoupled manner. Due to this decoupled network is entirely interpretable by design. Now, we first summarise the main results in the prior work by \cite{npk} who developed a dual view to understand the role of gates in DNNs with ReLUs, followed by the specific contributions in this paper.




 %Recent past has seen two paradigms namely `explainability' and `interpretabilty' to addres the issue of `black box'-ness. In the  `explaniablity' paradigm, one accepts as a fact that complex machine learning tasks might require complex black box models, however, resorts to \emph{post-hoc} \emph{explaination} of the \emph{decisions} of a DNN by building simpler local models. In the `interpretability' paradigm, one builds entirely interpretable white box models in the first place, thereby eliminating the need for \emph{post-hoc} explanations. 



\subsection{Organisation and Contribution}
We present the preliminaries of the dual view in \Cref{sec:prior}, followed by the novel contributions of the paper in rest of the sections which we briefly list below.

\begin{itemize}
\item \Cref{sec:fc} We 
\end{itemize}





\bibliography{refs}
\bibliographystyle{iclr2022_conference}

\appendix
\section{Appendix}
You may include other additional sections here.

\end{document}

