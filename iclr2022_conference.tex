\documentclass{article} % For LaTeX2e
\usepackage{iclr2022_conference,times}
\usepackage{hyperref}
\usepackage{url}
%!TEX root =  autocontgrlp.tex
\usepackage{savesym}
\usepackage{multicol}
\usepackage{multirow}
\usepackage{dsfont}
\usepackage{tabularx}
%\setlength{\marginparwidth}{13mm}
\usepackage[textsize=tiny]{todonotes}
%\usepackage[disable]{todonotes}
\newcommand{\todoch}[2][]{\todo[color=blue!20!white,#1]{C: #2}}
%\usepackage{enumitem}
%\usepackage[fleqn]{amsmath}
\usepackage{amsmath}
\usepackage{mathtools}
\usepackage{graphicx}
\usepackage{times}
\usepackage{helvet}
\usepackage{courier}
\usepackage{paralist}
\usepackage{latexsym}
\usepackage{url}
\usepackage[all]{xy}
\usepackage{amsmath}
\usepackage{amssymb}
\usepackage{amsthm}
\usepackage{nccmath} % mfrac
\usepackage{wrapfig}
\usepackage{comment}
%\usepackage{enumitem}
\usepackage{paralist}
\usepackage{xcolor}
%\usepackage[colorlinks=true,linkcolor=blue,citecolor=purple]{hyperref}
%\usepackage{hyperref}
\usepackage{graphicx}
\usepackage{pifont}
%\usepackage{algorithm}
%\usepackage{algorithmic}
%\usepackage{pseudocode}
%\usepackage{algpseudocode}
\usepackage{savesym}
\savesymbol{AND}
\usepackage{xspace}
\usepackage{tikz}
\usepackage{pgfplots}
\usepackage{pgf}
\usepackage{algorithm}
\usepackage{algorithmic}
\usepackage{xspace}
\usepackage{comment}
\usepackage{placeins}
\usepackage{cancel}
\usepackage{changepage}
\usepackage{bbm}
\usepackage{pgfplots}
\usepackage{filecontents}
\setcounter{MaxMatrixCols}{32}
\usepackage{placeins}
\usepackage{esvect}

\defcitealias{ch2020neural}{LS2020}
\definecolor{electricpurple}{rgb}{0.75, 0.0, 1.0}
\definecolor{darkbrown}{rgb}{0.4, 0.26, 0.13}
\definecolor{cocoabrown}{rgb}{0.82, 0.41, 0.12}
\definecolor{copper}{rgb}{0.72, 0.45, 0.2}
\definecolor{chocolate}{rgb}{0.82, 0.41, 0.12}
\usepackage{bm}

\usepackage[capitalize]{cleveref}
%\usepackage{caption}

%\usepackage{style/ssltr}
%\usepackage{style/macros}
\if0
\usetikzlibrary{intersections}
\usetikzlibrary{arrows,calc,fit,patterns,plotmarks,shapes.geometric,shapes.misc,shapes.symbols,   shapes.arrows,   shapes.callouts,   shapes.multipart,   shapes.gates.logic.US,   shapes.gates.logic.IEC,   er,   automata,   backgrounds,   chains,   topaths,   trees,   petri,   mindmap,   matrix,   calendar,   folding, fadings,   through,   positioning,   scopes,   decorations.fractals,   decorations.shapes,   decorations.text,   decorations.pathmorphing,   decorations.pathreplacing,   decorations.footprints,   decorations.markings, shadows,circuits}
\tikzstyle{decision}=[diamond,draw]
\tikzstyle{line}=[draw]
\tikzstyle{elli}=[draw,ellipse]
\tikzstyle{arrow} = [thick]
\fi
%\usepackage{subfig}

\newcommand{\mb}{\mbox{ }}
\newcommand{\one}{\mathbf{1}}
\newcommand{\zero}{\mathbf{0}}
\newcommand{\nn}{\nonumber}


\newcommand{\rsa}{\rightsquigarrow}
\newcommand{\ra}{\rightarrow}

\newcommand{\inrdnet}{\in\R^{d_{\text{net}}}}
\newcommand{\inrdin}{\in\R^{d_{\text{in}}}}
\newcommand{\lmin}[1]{\lambda_{\min}\left(#1\right)}

%VTK
\newcommand{\V}{\mathcal{V}}
\newcommand{\F}{\mathcal{F}}
%Indices
\newcommand{\I}{\mathcal{I}}
\newcommand{\Ifeat}{\mathcal{I}^{\text{f}}}
\newcommand{\Iker}{\mathcal{I}^{\text{ker}}}
\newcommand{\Iconv}{\mathcal{I}^{\text{cv}}}
\newcommand{\Ifc}{\mathcal{I}^{\text{fc}}}
\newcommand{\Iwidth}{\mathcal{I}^{\text{w}}}


%Features
\newcommand{\psiv}{\psi^{\text{v}}}
\newcommand{\psif}{\psi^{\text{f}}}


%Kernel
\newcommand{\kv}{K^{\text{v}}}
\newcommand{\kf}{K^{\text{f}}}
\newcommand{\kc}{K^{\text{cross}}}


\newcommand{\yc}{y^{\text{conv}}}
\newcommand{\din}{d_{\text{in}}}
\newcommand{\dnet}{d_{\text{net}}}
\newcommand{\dfnet}{d^{\text{f}}_{\text{net}}}
\newcommand{\dvnet}{d^{\text{v}}_{\text{net}}}

%Expectation
%\newcommand{\E}{\mathbb{E}}
%\newcommand{\E}[1]{\mathbb{E}\left[#1\right]}

%Weights
\newcommand{\tv}{{\theta^{\text{V}}}}

\newcommand{\Tdgn}{\Theta^{\text{DGN}}}
\newcommand{\Tb}{{\Theta}}
\newcommand{\Tv}{{\Theta}^{\text{v}}}
\newcommand{\Tg}{{\Theta}^{\text{f}}}
\newcommand{\tg}{\theta^{\text{f}}}
\newcommand{\Tt}{\Theta}
\newcommand{\Tf}{\Theta^{\textrm{f}}}

%Gates
\newcommand{\G}{\mathcal{G}}
\newcommand{\relu}{\gamma_{\text{r}}}
\newcommand{\softrelu}{\gamma_{\text{sr}}}
\newcommand{\ut}{U_{\text{tot}}}

%Norms
\newcommand{\norm}[1]{\|#1\|}
\newcommand{\inorm}[1]{\|#1\|_{\infty}}
\newcommand{\snorm}[1]{\left\|#1\right\|}
\newcommand{\sinorm}[1]{\left\|#1\right\|_{\infty}}


%Layers
\newcommand{\dc}{d_{\text{cv}}}
\newcommand{\dconv}{d_{\text{cv}}}

\newcommand{\dfc}{d_{\text{fc}}}
\newcommand{\dres}{d_{\text{rs}}}
\newcommand{\dblock}{d_{\text{blk}}}
\newcommand{\wc}{w_{\text{cv}}}
\newcommand{\wfc}{w_{\text{fc}}}
\newcommand{\wker}{w_{\text{ker}}}
\newcommand{\wconv}{w_{\text{cv}}}


\newcommand{\ifin}{i_{\text{fin}}}
\newcommand{\ifout}{i_{\text{fout}}}
\newcommand{\icin}{i_{\text{cv}}}
\newcommand{\iin}{i_{\text{in}}}
\newcommand{\iout}{i_{\text{out}}}
\newcommand{\iker}{i_{\text{ker}}}
\newcommand{\iu}{i_{\text{u}}}
\newcommand{\dgate}{d_{\text{gt}}}
\newcommand{\dnorm}{d_{\text{nrm}}}
\newcommand{\dweight}{d_{\text{wt}}}


\newcommand{\Igate}{\mathcal{I}_{\text{gt}}}
\newcommand{\Inorm}{\mathcal{I}_{\text{nrm}}}
\newcommand{\Iweight}{\mathcal{I}_{\text{wt}}}

\newcommand{\xv}{x^{\text{v}}}
\newcommand{\xf}{x^{\text{f}}}

\newcommand{\nt}{\nabla_{\Theta}}

\newcommand{\Pres}{P^{\text{res}}}
\newcommand{\Pfc}{P^{\text{fc}}}
\newcommand{\Pcnn}{P^{\text{cnn}}}

\newcommand{\eqdef}{\stackrel{\Delta}{=}}

%Constants
\newcommand{\cfc}{C_{\text{fc}}}
\newcommand{\ccnn}{C_{\text{cv}}}
\newcommand{\cres}{C_{\text{rs}}}
\newcommand{\cscale}{c_{\text{scale}}}
\newcommand{\sigfc}{\sigma_{\text{fc}}}
\newcommand{\sigcnn}{\sigma_{\text{cv}}}

\newcommand{\bfc}{\beta_{\text{fc}}}
\newcommand{\bcnn}{\beta_{\text{cv}}}
\newcommand{\bres}{\beta_{\text{res}}}

\newcommand{\keywords}[1]{{\bf Keywords: } #1\par}
%\newenvironment{proof}{{\bf Proof:} }{}
\newtheorem{theorem}{Theorem}[section]
\newtheorem{lemma}{Lemma}[section]
\newtheorem{claim}[theorem]{Claim}
\newtheorem{proposition}{Proposition}[section]
\newtheorem{corollary}[theorem]{Corollary}
\newtheorem{assumption}{Assumption}[section]
\newtheorem{definition}{Definition}[section]
\newtheorem{remark}{Remark}[section]
\newtheorem{identity}{Identity}[section]
\newtheorem{example}{Example}[section]
\newtheorem{note}{Note}[section]
\newtheorem{notation}{Notation}[section]
\newcommand{\alert}[1]{\textcolor{red}{#1}} 
\newcommand{\J}{\mathcal{J}}
\newcommand{\ip}[1]{\langle #1\rangle}

\def\v{\mathbf{v}}
\def\r{\mathbf{r}}
\def\p{\mathbf{p}}
\def\q{\mathbf{q}}
%\def\R{\mathbb{R}}
\def\Re{\mathbb{R}}
\def\Z{\mathbb{Z}}
\def\P{\mathcal{P}}
\def\S{\mathcal{S}}
\def\A{\mathcal{A}}
\def\X{\mathcal{X}}
\def\U{\mathcal{U}}

%\newcommand{\B}{\mathcal{B}}


\newcommand{\ith}[2][th]{$#2^{\text{#1}}$}
\newcommand{\us}[2]{\underset{#2}{#1}~}
\newcounter{subequation}[equation]
\newcommand{\thesubequationonly}{\alph{subequation}}
\renewcommand{\thesubequation}{\text{\theequation(\thesubequationonly)}}
\newcommand{\subequationitem}{\refstepcounter{subequation}(\thesubequationonly)\thinspace}

\def\mathdisplay#1{%
  \ifmmode \@badmath
  \else
    $\def\@currenvir{#1}%
    \let\dspbrk@context\z@
    \let\tag\tag@in@display \SK@equationtrue %\let\label\label@in@display
    \global\let\df@label\@empty \global\let\df@tag\@empty
    \global\tag@false
    \let\mathdisplay@push\mathdisplay@@push
    \let\mathdisplay@pop\mathdisplay@@pop
    \if@fleqn
      \edef\restore@hfuzz{\hfuzz\the\hfuzz\relax}%
      \hfuzz\maxdimen
      \setbox\z@\hbox to\displaywidth\bgroup
        \let\split@warning\relax \restore@hfuzz
        \everymath\@emptytoks \m@th $\displaystyle
    \fi
%   \fi
}


% Optional math commands from https://github.com/goodfeli/dlbook_notation.
%%%%% NEW MATH DEFINITIONS %%%%%

\usepackage{amsmath,amsfonts,bm}

% Mark sections of captions for referring to divisions of figures
\newcommand{\figleft}{{\em (Left)}}
\newcommand{\figcenter}{{\em (Center)}}
\newcommand{\figright}{{\em (Right)}}
\newcommand{\figtop}{{\em (Top)}}
\newcommand{\figbottom}{{\em (Bottom)}}
\newcommand{\captiona}{{\em (a)}}
\newcommand{\captionb}{{\em (b)}}
\newcommand{\captionc}{{\em (c)}}
\newcommand{\captiond}{{\em (d)}}

% Highlight a newly defined term
\newcommand{\newterm}[1]{{\bf #1}}


% Figure reference, lower-case.
\def\figref#1{figure~\ref{#1}}
% Figure reference, capital. For start of sentence
\def\Figref#1{Figure~\ref{#1}}
\def\twofigref#1#2{figures \ref{#1} and \ref{#2}}
\def\quadfigref#1#2#3#4{figures \ref{#1}, \ref{#2}, \ref{#3} and \ref{#4}}
% Section reference, lower-case.
\def\secref#1{section~\ref{#1}}
% Section reference, capital.
\def\Secref#1{Section~\ref{#1}}
% Reference to two sections.
\def\twosecrefs#1#2{sections \ref{#1} and \ref{#2}}
% Reference to three sections.
\def\secrefs#1#2#3{sections \ref{#1}, \ref{#2} and \ref{#3}}
% Reference to an equation, lower-case.
\def\eqref#1{equation~\ref{#1}}
% Reference to an equation, upper case
\def\Eqref#1{Equation~\ref{#1}}
% A raw reference to an equation---avoid using if possible
\def\plaineqref#1{\ref{#1}}
% Reference to a chapter, lower-case.
\def\chapref#1{chapter~\ref{#1}}
% Reference to an equation, upper case.
\def\Chapref#1{Chapter~\ref{#1}}
% Reference to a range of chapters
\def\rangechapref#1#2{chapters\ref{#1}--\ref{#2}}
% Reference to an algorithm, lower-case.
\def\algref#1{algorithm~\ref{#1}}
% Reference to an algorithm, upper case.
\def\Algref#1{Algorithm~\ref{#1}}
\def\twoalgref#1#2{algorithms \ref{#1} and \ref{#2}}
\def\Twoalgref#1#2{Algorithms \ref{#1} and \ref{#2}}
% Reference to a part, lower case
\def\partref#1{part~\ref{#1}}
% Reference to a part, upper case
\def\Partref#1{Part~\ref{#1}}
\def\twopartref#1#2{parts \ref{#1} and \ref{#2}}

\def\ceil#1{\lceil #1 \rceil}
\def\floor#1{\lfloor #1 \rfloor}
\def\1{\bm{1}}
\newcommand{\train}{\mathcal{D}}
\newcommand{\valid}{\mathcal{D_{\mathrm{valid}}}}
\newcommand{\test}{\mathcal{D_{\mathrm{test}}}}

\def\eps{{\epsilon}}


% Random variables
\def\reta{{\textnormal{$\eta$}}}
\def\ra{{\textnormal{a}}}
\def\rb{{\textnormal{b}}}
\def\rc{{\textnormal{c}}}
\def\rd{{\textnormal{d}}}
\def\re{{\textnormal{e}}}
\def\rf{{\textnormal{f}}}
\def\rg{{\textnormal{g}}}
\def\rh{{\textnormal{h}}}
\def\ri{{\textnormal{i}}}
\def\rj{{\textnormal{j}}}
\def\rk{{\textnormal{k}}}
\def\rl{{\textnormal{l}}}
% rm is already a command, just don't name any random variables m
\def\rn{{\textnormal{n}}}
\def\ro{{\textnormal{o}}}
\def\rp{{\textnormal{p}}}
\def\rq{{\textnormal{q}}}
\def\rr{{\textnormal{r}}}
\def\rs{{\textnormal{s}}}
\def\rt{{\textnormal{t}}}
\def\ru{{\textnormal{u}}}
\def\rv{{\textnormal{v}}}
\def\rw{{\textnormal{w}}}
\def\rx{{\textnormal{x}}}
\def\ry{{\textnormal{y}}}
\def\rz{{\textnormal{z}}}

% Random vectors
\def\rvepsilon{{\mathbf{\epsilon}}}
\def\rvtheta{{\mathbf{\theta}}}
\def\rva{{\mathbf{a}}}
\def\rvb{{\mathbf{b}}}
\def\rvc{{\mathbf{c}}}
\def\rvd{{\mathbf{d}}}
\def\rve{{\mathbf{e}}}
\def\rvf{{\mathbf{f}}}
\def\rvg{{\mathbf{g}}}
\def\rvh{{\mathbf{h}}}
\def\rvu{{\mathbf{i}}}
\def\rvj{{\mathbf{j}}}
\def\rvk{{\mathbf{k}}}
\def\rvl{{\mathbf{l}}}
\def\rvm{{\mathbf{m}}}
\def\rvn{{\mathbf{n}}}
\def\rvo{{\mathbf{o}}}
\def\rvp{{\mathbf{p}}}
\def\rvq{{\mathbf{q}}}
\def\rvr{{\mathbf{r}}}
\def\rvs{{\mathbf{s}}}
\def\rvt{{\mathbf{t}}}
\def\rvu{{\mathbf{u}}}
\def\rvv{{\mathbf{v}}}
\def\rvw{{\mathbf{w}}}
\def\rvx{{\mathbf{x}}}
\def\rvy{{\mathbf{y}}}
\def\rvz{{\mathbf{z}}}

% Elements of random vectors
\def\erva{{\textnormal{a}}}
\def\ervb{{\textnormal{b}}}
\def\ervc{{\textnormal{c}}}
\def\ervd{{\textnormal{d}}}
\def\erve{{\textnormal{e}}}
\def\ervf{{\textnormal{f}}}
\def\ervg{{\textnormal{g}}}
\def\ervh{{\textnormal{h}}}
\def\ervi{{\textnormal{i}}}
\def\ervj{{\textnormal{j}}}
\def\ervk{{\textnormal{k}}}
\def\ervl{{\textnormal{l}}}
\def\ervm{{\textnormal{m}}}
\def\ervn{{\textnormal{n}}}
\def\ervo{{\textnormal{o}}}
\def\ervp{{\textnormal{p}}}
\def\ervq{{\textnormal{q}}}
\def\ervr{{\textnormal{r}}}
\def\ervs{{\textnormal{s}}}
\def\ervt{{\textnormal{t}}}
\def\ervu{{\textnormal{u}}}
\def\ervv{{\textnormal{v}}}
\def\ervw{{\textnormal{w}}}
\def\ervx{{\textnormal{x}}}
\def\ervy{{\textnormal{y}}}
\def\ervz{{\textnormal{z}}}

% Random matrices
\def\rmA{{\mathbf{A}}}
\def\rmB{{\mathbf{B}}}
\def\rmC{{\mathbf{C}}}
\def\rmD{{\mathbf{D}}}
\def\rmE{{\mathbf{E}}}
\def\rmF{{\mathbf{F}}}
\def\rmG{{\mathbf{G}}}
\def\rmH{{\mathbf{H}}}
\def\rmI{{\mathbf{I}}}
\def\rmJ{{\mathbf{J}}}
\def\rmK{{\mathbf{K}}}
\def\rmL{{\mathbf{L}}}
\def\rmM{{\mathbf{M}}}
\def\rmN{{\mathbf{N}}}
\def\rmO{{\mathbf{O}}}
\def\rmP{{\mathbf{P}}}
\def\rmQ{{\mathbf{Q}}}
\def\rmR{{\mathbf{R}}}
\def\rmS{{\mathbf{S}}}
\def\rmT{{\mathbf{T}}}
\def\rmU{{\mathbf{U}}}
\def\rmV{{\mathbf{V}}}
\def\rmW{{\mathbf{W}}}
\def\rmX{{\mathbf{X}}}
\def\rmY{{\mathbf{Y}}}
\def\rmZ{{\mathbf{Z}}}

% Elements of random matrices
\def\ermA{{\textnormal{A}}}
\def\ermB{{\textnormal{B}}}
\def\ermC{{\textnormal{C}}}
\def\ermD{{\textnormal{D}}}
\def\ermE{{\textnormal{E}}}
\def\ermF{{\textnormal{F}}}
\def\ermG{{\textnormal{G}}}
\def\ermH{{\textnormal{H}}}
\def\ermI{{\textnormal{I}}}
\def\ermJ{{\textnormal{J}}}
\def\ermK{{\textnormal{K}}}
\def\ermL{{\textnormal{L}}}
\def\ermM{{\textnormal{M}}}
\def\ermN{{\textnormal{N}}}
\def\ermO{{\textnormal{O}}}
\def\ermP{{\textnormal{P}}}
\def\ermQ{{\textnormal{Q}}}
\def\ermR{{\textnormal{R}}}
\def\ermS{{\textnormal{S}}}
\def\ermT{{\textnormal{T}}}
\def\ermU{{\textnormal{U}}}
\def\ermV{{\textnormal{V}}}
\def\ermW{{\textnormal{W}}}
\def\ermX{{\textnormal{X}}}
\def\ermY{{\textnormal{Y}}}
\def\ermZ{{\textnormal{Z}}}

% Vectors
\def\vzero{{\bm{0}}}
\def\vone{{\bm{1}}}
\def\vmu{{\bm{\mu}}}
\def\vtheta{{\bm{\theta}}}
\def\va{{\bm{a}}}
\def\vb{{\bm{b}}}
\def\vc{{\bm{c}}}
\def\vd{{\bm{d}}}
\def\ve{{\bm{e}}}
\def\vf{{\bm{f}}}
\def\vg{{\bm{g}}}
\def\vh{{\bm{h}}}
\def\vi{{\bm{i}}}
\def\vj{{\bm{j}}}
\def\vk{{\bm{k}}}
\def\vl{{\bm{l}}}
\def\vm{{\bm{m}}}
\def\vn{{\bm{n}}}
\def\vo{{\bm{o}}}
\def\vp{{\bm{p}}}
\def\vq{{\bm{q}}}
\def\vr{{\bm{r}}}
\def\vs{{\bm{s}}}
\def\vt{{\bm{t}}}
\def\vu{{\bm{u}}}
\def\vv{{\bm{v}}}
\def\vw{{\bm{w}}}
\def\vx{{\bm{x}}}
\def\vy{{\bm{y}}}
\def\vz{{\bm{z}}}

% Elements of vectors
\def\evalpha{{\alpha}}
\def\evbeta{{\beta}}
\def\evepsilon{{\epsilon}}
\def\evlambda{{\lambda}}
\def\evomega{{\omega}}
\def\evmu{{\mu}}
\def\evpsi{{\psi}}
\def\evsigma{{\sigma}}
\def\evtheta{{\theta}}
\def\eva{{a}}
\def\evb{{b}}
\def\evc{{c}}
\def\evd{{d}}
\def\eve{{e}}
\def\evf{{f}}
\def\evg{{g}}
\def\evh{{h}}
\def\evi{{i}}
\def\evj{{j}}
\def\evk{{k}}
\def\evl{{l}}
\def\evm{{m}}
\def\evn{{n}}
\def\evo{{o}}
\def\evp{{p}}
\def\evq{{q}}
\def\evr{{r}}
\def\evs{{s}}
\def\evt{{t}}
\def\evu{{u}}
\def\evv{{v}}
\def\evw{{w}}
\def\evx{{x}}
\def\evy{{y}}
\def\evz{{z}}

% Matrix
\def\mA{{\bm{A}}}
\def\mB{{\bm{B}}}
\def\mC{{\bm{C}}}
\def\mD{{\bm{D}}}
\def\mE{{\bm{E}}}
\def\mF{{\bm{F}}}
\def\mG{{\bm{G}}}
\def\mH{{\bm{H}}}
\def\mI{{\bm{I}}}
\def\mJ{{\bm{J}}}
\def\mK{{\bm{K}}}
\def\mL{{\bm{L}}}
\def\mM{{\bm{M}}}
\def\mN{{\bm{N}}}
\def\mO{{\bm{O}}}
\def\mP{{\bm{P}}}
\def\mQ{{\bm{Q}}}
\def\mR{{\bm{R}}}
\def\mS{{\bm{S}}}
\def\mT{{\bm{T}}}
\def\mU{{\bm{U}}}
\def\mV{{\bm{V}}}
\def\mW{{\bm{W}}}
\def\mX{{\bm{X}}}
\def\mY{{\bm{Y}}}
\def\mZ{{\bm{Z}}}
\def\mBeta{{\bm{\beta}}}
\def\mPhi{{\bm{\Phi}}}
\def\mLambda{{\bm{\Lambda}}}
\def\mSigma{{\bm{\Sigma}}}

% Tensor
\DeclareMathAlphabet{\mathsfit}{\encodingdefault}{\sfdefault}{m}{sl}
\SetMathAlphabet{\mathsfit}{bold}{\encodingdefault}{\sfdefault}{bx}{n}
\newcommand{\tens}[1]{\bm{\mathsfit{#1}}}
\def\tA{{\tens{A}}}
\def\tB{{\tens{B}}}
\def\tC{{\tens{C}}}
\def\tD{{\tens{D}}}
\def\tE{{\tens{E}}}
\def\tF{{\tens{F}}}
\def\tG{{\tens{G}}}
\def\tH{{\tens{H}}}
\def\tI{{\tens{I}}}
\def\tJ{{\tens{J}}}
\def\tK{{\tens{K}}}
\def\tL{{\tens{L}}}
\def\tM{{\tens{M}}}
\def\tN{{\tens{N}}}
\def\tO{{\tens{O}}}
\def\tP{{\tens{P}}}
\def\tQ{{\tens{Q}}}
\def\tR{{\tens{R}}}
\def\tS{{\tens{S}}}
\def\tT{{\tens{T}}}
\def\tU{{\tens{U}}}
\def\tV{{\tens{V}}}
\def\tW{{\tens{W}}}
\def\tX{{\tens{X}}}
\def\tY{{\tens{Y}}}
\def\tZ{{\tens{Z}}}


% Graph
\def\gA{{\mathcal{A}}}
\def\gB{{\mathcal{B}}}
\def\gC{{\mathcal{C}}}
\def\gD{{\mathcal{D}}}
\def\gE{{\mathcal{E}}}
\def\gF{{\mathcal{F}}}
\def\gG{{\mathcal{G}}}
\def\gH{{\mathcal{H}}}
\def\gI{{\mathcal{I}}}
\def\gJ{{\mathcal{J}}}
\def\gK{{\mathcal{K}}}
\def\gL{{\mathcal{L}}}
\def\gM{{\mathcal{M}}}
\def\gN{{\mathcal{N}}}
\def\gO{{\mathcal{O}}}
\def\gP{{\mathcal{P}}}
\def\gQ{{\mathcal{Q}}}
\def\gR{{\mathcal{R}}}
\def\gS{{\mathcal{S}}}
\def\gT{{\mathcal{T}}}
\def\gU{{\mathcal{U}}}
\def\gV{{\mathcal{V}}}
\def\gW{{\mathcal{W}}}
\def\gX{{\mathcal{X}}}
\def\gY{{\mathcal{Y}}}
\def\gZ{{\mathcal{Z}}}

% Sets
\def\sA{{\mathbb{A}}}
\def\sB{{\mathbb{B}}}
\def\sC{{\mathbb{C}}}
\def\sD{{\mathbb{D}}}
% Don't use a set called E, because this would be the same as our symbol
% for expectation.
\def\sF{{\mathbb{F}}}
\def\sG{{\mathbb{G}}}
\def\sH{{\mathbb{H}}}
\def\sI{{\mathbb{I}}}
\def\sJ{{\mathbb{J}}}
\def\sK{{\mathbb{K}}}
\def\sL{{\mathbb{L}}}
\def\sM{{\mathbb{M}}}
\def\sN{{\mathbb{N}}}
\def\sO{{\mathbb{O}}}
\def\sP{{\mathbb{P}}}
\def\sQ{{\mathbb{Q}}}
\def\sR{{\mathbb{R}}}
\def\sS{{\mathbb{S}}}
\def\sT{{\mathbb{T}}}
\def\sU{{\mathbb{U}}}
\def\sV{{\mathbb{V}}}
\def\sW{{\mathbb{W}}}
\def\sX{{\mathbb{X}}}
\def\sY{{\mathbb{Y}}}
\def\sZ{{\mathbb{Z}}}

% Entries of a matrix
\def\emLambda{{\Lambda}}
\def\emA{{A}}
\def\emB{{B}}
\def\emC{{C}}
\def\emD{{D}}
\def\emE{{E}}
\def\emF{{F}}
\def\emG{{G}}
\def\emH{{H}}
\def\emI{{I}}
\def\emJ{{J}}
\def\emK{{K}}
\def\emL{{L}}
\def\emM{{M}}
\def\emN{{N}}
\def\emO{{O}}
\def\emP{{P}}
\def\emQ{{Q}}
\def\emR{{R}}
\def\emS{{S}}
\def\emT{{T}}
\def\emU{{U}}
\def\emV{{V}}
\def\emW{{W}}
\def\emX{{X}}
\def\emY{{Y}}
\def\emZ{{Z}}
\def\emSigma{{\Sigma}}

% entries of a tensor
% Same font as tensor, without \bm wrapper
\newcommand{\etens}[1]{\mathsfit{#1}}
\def\etLambda{{\etens{\Lambda}}}
\def\etA{{\etens{A}}}
\def\etB{{\etens{B}}}
\def\etC{{\etens{C}}}
\def\etD{{\etens{D}}}
\def\etE{{\etens{E}}}
\def\etF{{\etens{F}}}
\def\etG{{\etens{G}}}
\def\etH{{\etens{H}}}
\def\etI{{\etens{I}}}
\def\etJ{{\etens{J}}}
\def\etK{{\etens{K}}}
\def\etL{{\etens{L}}}
\def\etM{{\etens{M}}}
\def\etN{{\etens{N}}}
\def\etO{{\etens{O}}}
\def\etP{{\etens{P}}}
\def\etQ{{\etens{Q}}}
\def\etR{{\etens{R}}}
\def\etS{{\etens{S}}}
\def\etT{{\etens{T}}}
\def\etU{{\etens{U}}}
\def\etV{{\etens{V}}}
\def\etW{{\etens{W}}}
\def\etX{{\etens{X}}}
\def\etY{{\etens{Y}}}
\def\etZ{{\etens{Z}}}

% The true underlying data generating distribution
\newcommand{\pdata}{p_{\rm{data}}}
% The empirical distribution defined by the training set
\newcommand{\ptrain}{\hat{p}_{\rm{data}}}
\newcommand{\Ptrain}{\hat{P}_{\rm{data}}}
% The model distribution
\newcommand{\pmodel}{p_{\rm{model}}}
\newcommand{\Pmodel}{P_{\rm{model}}}
\newcommand{\ptildemodel}{\tilde{p}_{\rm{model}}}
% Stochastic autoencoder distributions
\newcommand{\pencode}{p_{\rm{encoder}}}
\newcommand{\pdecode}{p_{\rm{decoder}}}
\newcommand{\precons}{p_{\rm{reconstruct}}}

\newcommand{\laplace}{\mathrm{Laplace}} % Laplace distribution

\newcommand{\E}{\mathbb{E}}
\newcommand{\Ls}{\mathcal{L}}
\newcommand{\R}{\mathbb{R}}
\newcommand{\emp}{\tilde{p}}
\newcommand{\lr}{\alpha}
\newcommand{\reg}{\lambda}
\newcommand{\rect}{\mathrm{rectifier}}
\newcommand{\softmax}{\mathrm{softmax}}
\newcommand{\sigmoid}{\sigma}
\newcommand{\softplus}{\zeta}
\newcommand{\KL}{D_{\mathrm{KL}}}
\newcommand{\Var}{\mathrm{Var}}
\newcommand{\standarderror}{\mathrm{SE}}
\newcommand{\Cov}{\mathrm{Cov}}
% Wolfram Mathworld says $L^2$ is for function spaces and $\ell^2$ is for vectors
% But then they seem to use $L^2$ for vectors throughout the site, and so does
% wikipedia.
\newcommand{\normlzero}{L^0}
\newcommand{\normlone}{L^1}
\newcommand{\normltwo}{L^2}
\newcommand{\normlp}{L^p}
\newcommand{\normmax}{L^\infty}

\newcommand{\parents}{Pa} % See usage in notation.tex. Chosen to match Daphne's book.

\DeclareMathOperator*{\argmax}{arg\,max}
\DeclareMathOperator*{\argmin}{arg\,min}

\DeclareMathOperator{\sign}{sign}
\DeclareMathOperator{\Tr}{Tr}
\let\ab\allowbreak






\title{Duality simplifies deep neural networks with rectified linear units}

% Authors must not appear in the submitted version. They should be hidden
% as long as the \iclrfinalcopy macro remains commented out below.
% Non-anonymous submissions will be rejected without review.

\author{Antiquus S.~Hippocampus, Natalia Cerebro \& Amelie P. Amygdale \thanks{ Use footnote for providing further information
about author (webpage, alternative address)---\emph{not} for acknowledging
funding agencies.  Funding acknowledgements go at the end of the paper.} \\
Department of Computer Science\\
Cranberry-Lemon University\\
Pittsburgh, PA 15213, USA \\
\texttt{\{hippo,brain,jen\}@cs.cranberry-lemon.edu} \\
\And
Ji Q. Ren \& Yevgeny LeNet \\
Department of Computational Neuroscience \\
University of the Witwatersrand \\
Joburg, South Africa \\
\texttt{\{robot,net\}@wits.ac.za} \\
\AND
Coauthor \\
Affiliation \\
Address \\
\texttt{email}
}

% The \author macro works with any number of authors. There are two commands
% used to separate the names and addresses of multiple authors: \And and \AND.
%
% Using \And between authors leaves it to \LaTeX{} to determine where to break
% the lines. Using \AND forces a linebreak at that point. So, if \LaTeX{}
% puts 3 of 4 authors names on the first line, and the last on the second
% line, try using \AND instead of \And before the third author name.

\newcommand{\fix}{\marginpar{FIX}}
\newcommand{\new}{\marginpar{NEW}}

%\iclrfinalcopy % Uncomment for camera-ready version, but NOT for submission.
\begin{document}



\maketitle

%\begin{comment}
\begin{abstract}
Despite their success deep neural networks (DNNs) are still largely considered as black boxes. The main issue is that the linear and non-linear operations are entangled in every layer, making it hard to interpret the hidden layer outputs. In this paper, we look at DNNs with rectified linear units (ReLUs), and focus on the gating property (`on/off' states) of the ReLUs. We extend the recently developed dual view in which the computation is broken path-wise to show that learning in the gates is more crucial, and learning the weights given the gates is characterised analytically via the so called \emph{neural path kernel} (NPK) which depends on inputs and gates. In this paper, we present novel results to show that convolution with global pooling and skip connection provide respectively rotational invariance and ensemble structure to the NPK. To address `black box'-ness, we propose a novel interpretable counterpart of DNNs with ReLUs namely deep linearly gated networks (DLGN): the pre-activations to the gates are generated by a deep linear network, and the gates are then applied as external masks to learn the weights in a different network. The DLGN is not an alternative architecture per se, but a disentanglement and an interpretable re-arrangement of the computations in a DNN with ReLUs. The DLGN disentangles the computations into two  `mathematically' interpretable linearities (i) the `primal' linearity between the input and the pre-activations in the gating network and (ii) the `dual' linearity in the path space in the weights network characterised by the NPK. We compare the performance of DNN, DGN and DLGN on CIFAR-10 and CIFAR-100 to show that, the DLGN recovers more than $83.5\%$ of the performance of state-of-the-art DNNs, i.e., while entanglement in the DNNs enable their improved performance,  the `disentangled and interpretable'  computations in the DLGN recovers most part of the performance. This brings us to an interesting question: `Is DLGN a universal spectral approximator?'%Finally, we use dual view to show that the degradation in the gates is the reason for degradation in the test performance due to upstream training with random labels (this was an open question in \cite{randlabel}).
%the DLGN counterparts of state of the art DNNs recover more than $83.5\%$ of the performance of the DNNs, which implies that while entanglement in the DNNs enable their improved performance,\
 \end{abstract}
%\end{comment}


\begin{comment}
\begin{abstract}
Despite their success deep neural networks (DNNs) are still largely considered as black boxes. The main issue is that the linear and non-linear operations are entangled in every layer, making it hard to interpret the hidden layer outputs. In this paper, we look at DNNs with rectified linear units (ReLUs), and focus on the gating property (`on/off' states) of the ReLUs. We extend the recently developed dual view in which the computation is broken path-wise to show that learning in the gates is more crucial, and learning the weights given the gates is characterised analytically via the so called \emph{neural path kernel} (NPK) which depends on inputs and gates. In this paper, we first extend the dual view to show that convolution with global pooling and skip connection provide rotational invariance and ensemble structure to NPK respectively. In order to address the issue of `black box'-ness, we propose a novel interpretable counterpart of DNNs with ReLUs namely deep linearly gated networks (DLGN): the pre-activations to the gates are generated by a deep linear network, and the gates are then applied as external masks to learn the weights in a different network. The DLGN disentangles the computations into two  `mathematically' interpretable linearities (i) the `primal' linearity between the input and the pre-activations, and (ii) the `dual' linearity in the path space interpreted via the NPK, and the role of gating is to \emph{lift} the primal to the dual. Our experiments on CIFAR-10 and CIFAR-100 show that the `disentangled and interpretable'  computations in the DLGN can recover more than $83.5\%$ of the performance of the DNNs. 
%the DLGN counterparts of state of the art DNNs recover more than $83.5\%$ of the performance of the DNNs, which implies that while entanglement in the DNNs enable their improved performance,\
 \end{abstract}


\begin{abstract}
We consider DNNs with rectified linear units (ReLUs). We focus on the gates, i.e., `on/off' states of the ReLUs. We build on dual view of \cite{npk} in which the computation is broken path-wise, where each path has both gates and weights. They showed that the input dependent gating patterns act like features, and characterised them analytically via the so called \emph{neural path kernel} (NPK) which depends on number of paths simultaneously `on' for a given input pair. New results in this paper deal with: (i) uncovering structural properties of NPK, (ii) demonstrating `layer-path' duality (iii) resolving an open question on training with random labels and (i) building comptetitve white box models.
%Firstly, we present new results that simplify and improve our understanding of DNNs with ReLUs. For this, we present three new theoretical results, using which we interepret the roles of activations, weights, width, depth, convolutions with global pooling and skip connections. 
Firstly, we show that the NPK has a product structure invariant to layer permutation and only depends on the correlation of the gates. We show that convolution with global pooling and skip connection provide rotational invariance and ensemble structure to NPK respectively. Secondly, we show that once the gates are obtained in a layer-by-layer manner, operations destroying the layer-by-layer structure such as permuting the layers, arbitrarily tiling and rotation of the gates and providing a constant input do not degrade performance, because, in all these operations, the correlation of gates is not lost. Thirdly, we use the dual view to resolve the open question related to degradation of test accuracy due to upstream training with random labels : degradation occurs because random labels affect the gates. Finally, we propose a novel model by modifying  DNNs with ReLUs: pre-activations to the gates are generated without any non-linear activations. Interpreted in the dual view, this novel model is entirely white box. We show white box models obtained by modifying VGG and a ResNet (in the proposed way) achieve greater than $90\%$ and close to $70\%$ on CIFAR-10 and CIFAR-100 respectively.  
%We present an improved and simplified understanding of deep neural network (DNNs) with rectified linear units (ReLUs), and use our understanding to build white box models. We focus on the gates, i.e., `on/off' states of the ReLUs. We build on prior work by \cite{npk} developed a dual view to showed that most information is in the gates, and captured the role of the gates analytically via the so called \emph{neural path kernel} (NPK). 
%In this paper, we show that interpreting via the dual view offers novel, fundamental, surprising and counter-intutive results eventually paving way for building white box models. %We achieve our aim by persuing two goals (i) conceptual: where our key objective is improved understanding and not necessarily to beat state of the art  and (ii) practical: where our key objective is to improve interpretability without significant loss with respect to state of the art. 
%FThe conceptual goal is to drive home the DNNs are interpretable in the dual view. 
%We reach our conceptual goal in two steps. 
%Firstly, we present new theoretical results by refining the prior work on dual view for the fully connected case, and extending it to cover the case of convolutions with global pooling and skip connections. The main highlight is that each layer has a base kernel measuring the correlation of gates, and the NPK is a Hadamard product of these base kernels and the input Gram matrix. We experimentally verify that operations destroying the layer-by-layer structure such as permuting the layers, arbitrarily tiling and rotation of the gates and providing a constant input do not degrade performance, because, in all these operations, the correlation of gates is not lost. We also show that upstream training with random labels degrades the gates and hence test accuracy even after downstream training with true labels (this was an open question). Finally, we propose a novel model by modifying  DNNs with ReLUs: we generate the pre-activation to the gates without any non-linear activations. Interpreted in the dual view, this novel model is entirely white box. We show white box models obtained by modifying VGG and a ResNet (in the proposed way) achieve greater than $90\%$ and close to $70\%$ on CIFAR-10 and CIFAR-100 respectively. %The empirical results here are of two kinds (i) conceptual: the key objective is improved understanding by verifying the theory and not necessarily to beat `state of the art' and (ii) practical: the key objective is to build white box models without significant loss with respect to `state of the art'.
%We present an improved and simplified understanding of  deep neural network (DNNs) with rectified linear units (ReLUs). In particular, we focus on the gating property (i.e., \emph{on/off} state) of ReLU, due to which, for each input there is an \emph{active/on} sub-network comprising of those gates which are \emph{on} and the weights between those gates. Recently, \cite{npk} developed a \emph{dual view} to separate the gates and the weights. They showed that most information is in the gates, and captured the role of the active sub-networks analytically via the so called \emph{neural path kernel} (NPK). In this paper, we simplify the NPK by expressing it explicitly in terms of the \emph{correlation of gates}, and derive the additional properties of NPK in the presence of convolutions and skip connections. The main highlight is that each layer has a base kernel measuring the correlation of gates, and the NPK is a Hadamard product of these base kernels and the input Gram matrix.  These show that the dual view is a natural way to think about the inner workings of DNNs with ReLUs. Finally, we propose a novel deep network (we call this \texttt{DGN-No-ACT}) wherein the gates are generated without any non-linear activations. This makes \texttt{DGN-No-ACT} a completely white box model. We show that \texttt{DGN-No-ACT} based on standard architectures achieve more than $90\%$ and close to $70\%$ on CIFAR-10 and CIFAR-100 respectively.


%Finally, we modify standard architectures (VGG19 and a ResNet) to yield two deep gated networks in which feature extraction is free of activations and is separate from the gates and the weights -- these achieve greater than $90\%$ test accuracy on CIFAR-10. 

%The other two theoretical results extend the dual view to cover the cases of convolutions with pooling and skip connections.
%Our main message is that the gates hold most useful information. 
%We present a simplified and improved understanding of the inner workings of deep neural networks (DNNs) with rectified linear units (ReLUs) by focussing on the gating (i.e., `on/off' states) of the ReLUs. We build on prior work by \cite{npk} which also focussed on the role of gates in DNNs with ReLUs.  Our main claim is that gates are indeed the most fundamental entities in such DNNs that hold most useful information. We provide theoretical basis for the claim and experimental justification. Based on this simplified understanding, we conceptualise a DNN with ReLU to have three functional components (i) gating, (ii) pre-activation generation and (iii) weights.  In a  DNN with ReLU these three functionalities are shared/entangled between its weights and activation. We propose a novel modification to disentangle these three components thereby making the deep network entirely interpretable. We show that applying this modification on standard state of the art DNNs makes them entirely interpretable without significant loss of performance.
%we propose a novel modification that yields an entirely interpretable deep network.
 
%We present an improved and simplified understanding of  deep neural network (DNNs) with rectified linear units (ReLUs) by focussing on the gating property (i.e., on/off state) of ReLU.  We build on the \emph{dual view} introduced by \cite{npk}. The key simplification is the claim that DNNs with ReLUs are characterised by the \emph{correlation of gates}. We verify this claim by showing that operations destroying the layer-by-layer structure such as permuting the layers, arbitrarily tiling and rotation of the gates and providing a constant input do not degrade performance, because, in all these operations, the correlation of gates is not lost. We then take up an open question related to the degradation of test accuracy due to upstream training with random labels for study. Using the dual view, we show that this degradation is attributed to the gates thereby demonstrating the importance of the role of gates and efficacy of the dual view in understanding DNNs with ReLUs. Based on our improved understanding, we propose a novel modification that improves `interpretability' :  here (i) feature extraction, (ii) gating and (iii) weights are decoupled. We show on standard architectures that this novel modification achieves greater than $90\%$ and close to $70\%$ test accuracies on CIFAR-10 and CIFAR-100 respectively while improving interpretability.

%We present an improved and simplified understanding of  deep neural network (DNNs) with rectified linear units (ReLUs). We build upon the dual view developed by \cite{npk}, wherein the computations are broken path-by-path as opposed to the primal view where computations are layer-by-layer. In the dual view, the output is expressed as an inner product of so called \emph{neural path features} (which encodes the input dependent computation) and so called \emph{neural path values} (which encode computations common across inputs). Our theoretical results refine the prior dual view for fully connected case, and extend it to cover the cases of convolutions with global pooling and skip connections. We also present empirical results which, are open and counter intuitive from the primal view, however, are reconciled in the dual view. These results suggest that dual view is the natural way to interpret DNNs with ReLUs. Finally, we propose a novel modified deep network wherein the neural path features are generated without using any non-linear activations (which call this network \texttt{DGN-No-ACT}). Here, features are not hidden, and are entirely interpretable in terms of standard `image processing' operations. This makes \texttt{DGN-No-ACT} a completely white box model. We show that \texttt{DGN-No-ACT} based on standard architectures achieve more than $90\%$ and close to $70\%$ on CIFAR-10 and CIFAR-100 respectively.

%We present an improved and simplified understanding of  deep neural network (DNNs) with rectified linear units (ReLUs) by focussing on the gating (i.e., `active/inactive' states) of the ReLUs. We build upon the dual view wherein the computation in a DNN is broken down into paths. Prior work on dual view (for fully connected case) showed that the most useful information in the gates, and is characterised by a so called \emph{neural path kernel} (NPK) which depends on the total number of active paths. Our theoretical results refine the prior dual view for fully connected case, and extend it to cover the cases of convolutions with global pooling and skip connections. The main highlight is that each layer has a base kernel measuring the correlation of gates, and the NPK is a Hadamard product of these base kernels and the input Gram matrix. To our knowledge, this is the simplest kernel in literature that analytically characterises the gates. We experimentally verify that operations destroying the layer-by-layer structure such as permuting the layers, arbitrarily tiling and rotation of the gates and providing a constant input do not degrade performance, because, in all these operations, the correlation of gates is not lost. We also show experimentally that (open question related to) the degradation in test accuracy due to upstream training with random labels is because of the degradation to gates, thereby reinforcing the importance of gates. %We also present empirical results which, are open and counter intuitive from the primal view, however, are reconciled in the dual view. These results suggest that dual view is the natural way to interpret DNNs with ReLUs. 
%Finally, we propose a novel modified deep network wherein the gates are generated without using any non-linear activations (which call this network \texttt{DGN-No-ACT}). Here, features are not hidden, and are entirely interpretable in terms of standard `image processing' operations. This makes \texttt{DGN-No-ACT} a completely white box model. We show that \texttt{DGN-No-ACT} based on standard architectures achieve more than $90\%$ and close to $70\%$ on CIFAR-10 and CIFAR-100 respectively.
%for a given pair of inputs is equal to the product of the inner product of the inputs and the sizes of the overlapping active sub-networks corresponding to the inputs.
\end{abstract}
\end{comment}







%Despite their success deep neural networks (DNNs) are still largely considered as black boxes. In this paper, we present a simplified and improved understanding of DNNs with rectified linear units (ReLUs). Unlike several methods which ‘explain’ the inner workings of DNNs by building simpler local models, our approach is to ‘dismantle’ the DNN with ReLU into separate components. In particular, we focus on the gating property (i.e., on/off states) of the ReLUs. We build on the prior work by Lakshminarayanan & Singh (2020), who developed a novel dual view, wherein the gates are dismantled from the weights. In this paper, we carry forward this dismantling, where the gates in a DNN are dismantled layer-by-layer, and finally the gates in a layer are dismantled unit-by-unit. The key simplification is in our main claim that gates are indeed the most fundamental entities in DNNs with ReLUs. We provide theoretical basis for this claim and justify the same in experiments. Based on our theory and experiments, we argue that a DNN with ReLU into three functionalities namely (i) gating (ii) pre-activation generation and (iii) weights. While a standard DNN with ReLU is such that these three functionalities are shared/entangled between its weights and activation, we propose a novel modification wherein we disentangle the three components.



\section{Introduction}

Despite their success deep neural networks (DNNs) are still largely considered as black boxes. In this paper, we present a simplified and improved understanding of DNNs with rectified linear units (ReLUs). Our approach is to identify and understand the \emph{functional components} in a DNN with ReLUs in a step by step manner.  Based on our understanding of such functional components, we finally propose a novel way to put them together in an entirely interpretable way thereby removing the `black box'-ness. Our work builds on top of prior work by \cite{npk} who developed a \emph{dual view} for fully connected DNNs with ReLUs. Our approach is in contrast to approaches that \emph{explain} the \emph{decisions} of a DNN by building simpler local models.


\subsection{Dual View: First step in understanding the role of gates}

The primal view is that computation in a DNN proceeds from the input to output in a layer-by-layer manner. In the dual view, the computation in a DNN is broken down into paths, and the output is a summation of the individual path contributions. For this purpose, the dual view exploits the gating property of ReLU, that is, the \emph{on/off} or \emph{active/inactive} states of the ReLUs. A path starts at an input node, passes through a weight and a ReLU in each layer until it reaches the output node. A path is active all the gates in that path active, and its contribution is equal to the product of the signal at the input node and the weights in the path. In the dual view, it is natural to think that depending on the input, some subset of the paths get activated, and computation from input to output is restricted to the `active' subnetwork consisting of such active paths. Thus, the gates are input dependent and the weights remain the same across inputs. However, weights are triggering the gates in the first place. So, in order to separately the understand of their roles, the gates are also physically separated from the weights in \emph{deep gated network} (DGN) setup, wherein, the input dependent gates are generated in a so called \emph{feature network} (which a DNN with ReLU whose sole  is  generation of the gates) and then gating signals are applied as external masks to a so called \emph{value network} which carries out the input to output computation. Having separated the gates and the weights, the role of gates and active subnetworks was investigated in theory and experiments.

\textbf{Neural Path Kernel (Theory).} A \emph{neural path kernel} (NPK) is defined and is equal to the \emph{Hadamard} product of the input Gram matrix and a correlation matrix which measures the overlap between the sub-networks active for the various input pairs. Prior results  \cite{arora2019exact,cao2019generalization,ntk} have shown the equivalence of an infinite width DNN trained using gradient descent and its corresponding \emph{neural tangent kernel} (NTK) matrix, the Gram matrix of the gradient of the network output with respect to the weights. It was shown that when the gates and weights are separated, with the gates being fixed and only the weights trained, under randomised initialisation, in the limit of infinite width, the NTK becomes equal to (but for a scalar term) the NPK. This equivalence between NTK and NPK analytically characterises the role of the active subnetworks.


\textbf{Gate Learning (Experiments).}  It was shown active subnetworks are learnt during training and it improves test accuracy.  Using the gates from a pre-trained DNN as external masks one can retrain the weights and match the test accuracy of the original pre-trained DNN with ReLU. 
%This implies that most useful input dependent information is in the gates. 
It was shown that if gates from a randomly initialised (instead of pre-trained) DNN are used as external masks, and the weights are trained, the test accuracy drops significantly. 
%This implies that gates are learnt when a DNN with ReLU is trained and such learning improves generalisation.

\subsection{Contribution and Organisation}
 
We pursue two goals (i) pedagogical: here the pursuit is not propose new methods to beat the state of the art, but to improve our understanding of basic functional components namely weights, activation, depth, width, convolutions with global pooling  and skip connection (ii) practical: here the pursuit is to build a white box model without significance performance loss with respect to state of the art. The pedagogical goal is to drive home the message that, even though the primal and dual views are mathematically equivalent, when compared to the primal, the dual view is a natural and simple way to interpret and understand the inner workings of DNNs with ReLUs. The pedagogical goal is achieved in the following two steps. 

Step 1 (\Cref{sec:fc}): We first theoretically investigate the structural properties of NPK by refining and extending the prior work on dual view. \Cref{th:fc} shows that \emph{correlation of gates is key}, and in particular that each layer is associated with a base kernel which measures the average correlation of gates, and the NPK is a \emph{Hadamard} product of the input Gram matrix and the base kernels. This implies that the role of ReLU is gating, the primal role of weights is to trigger the ReLUs, the role of width is averaging and the role of depth is to form a product kernel. In \Cref{th:conv} we show that the role of convolutions with global pooling is to provide rotational invariance to the NPK. In \Cref{th:res} we show that the role of skip connections is to provide an ensemble structure to NPK.

Step 2  (\Cref{sec:exp}): Having refined and extended the NPK in theory, we empirically investigate two interesting scenarios, first one is a counter intuitive setting and second one is an open question from \cite{randlabel}. We describe these below.

\quad$\bullet$ We destroy the layer by layer structure by permuting the gates or even tiling the gates of all layers and rotating them arbitrarily. We empirically observe that this does not cause performance to degrade at all. Since, it could be argued that the network still manages to recover in a layer by layer manner despite such layer permutations. In order to eliminate this argument, we shut off the input as well by making the input to the value network to be 1. While this is perhaps surprising and counter-intuitive to the primal view that progressively complicated structures are being in a layer by layer manner, however, from \Cref{th:fc} it is known that  correlation of gates is key and operations destroy the layer by layer structure do not destroy the correlation of the gates, and hence the empirical observations are reconciled readily in the dual view.

\quad$\bullet$ We show that upstream training with random labels followed by downstream training with true labels degrades test accuracy because the random labels affects the gates. This degradation of test accuracy was an open question in \cite{randlabel}.


In all results above, the gates were generated by a feature network which was a DNN with ReLUs.  Based on our understanding, we propose to replace the ReLUs with identity activations giving rise to a white box architecture called \texttt{DGN-NO-ACT}. Once the ReLU non-linearity is removed, the other operations such as convolution (which is  linear), pooling and batch norm (bias and scaling) are well understood and interpretable in `image processing' terms. 





%In this paper, we carry forward this dismantling, where the gates in a DNN are dismantled layer-by-layer, and finally the gates in a layer are dismantled unit-by-unit. The key simplification is in our main claim that gates are indeed the most fundamental entities in DNNs with ReLUs. We provide theoretical basis for this claim and justify the same in experiments. Based on our theory and experiments, we argue that a DNN with ReLU into three functionalities namely (i) gating (ii) pre-activation generation and (iii) weights. While a standard DNN with ReLU is such that these three functionalities are shared/entangled between its weights and activation,  we propose a novel modification wherein we disentangle the three components. 

% separate the functionalities to dedicated for (i) feature generation (without any hidden units), (ii) gating, and (iii) weights. 

%Due to this separation of functionalities network is entirely interpretable by design. Now, we first summarise the main results in the prior work by \cite{npk} who developed a dual view to understand the role of gates in DNNs with ReLUs, followed by the specific contributions in this paper.





%Our main claim is that gates are indeed the most fundamental entities in such DNNs. We provide theoretical basis for the claim and justify the same in experiments. Based on this claim, we propose a novel modification wherein the deep network has separate components to dedicated specifically to address (i) feature generation (without any hidden units), (ii) gating, and (iii) the weights in a decoupled manner. Due to this decoupled network is entirely interpretable by design. Now, we first summarise the main results in the prior work by \cite{npk} who developed a dual view to understand the role of gates in DNNs with ReLUs, followed by the specific contributions in this paper.





\bibliography{refs}
\bibliographystyle{iclr2022_conference}

\appendix
\section{Appendix}
You may include other additional sections here.

\end{document}

